\documentclass{article}

% set font encoding for PDFLaTeX or XeLaTeX
\usepackage{ifxetex}
\ifxetex
  \usepackage{fontspec}
\else
  \usepackage[T1]{fontenc}
  \usepackage[utf8]{inputenc}
  \usepackage{lmodern}
  \usepackage{graphicx}
  \usepackage{wrapfig}
  \usepackage{float}
\fi

% used in maketitle
\title{Reporte de Actividad 1}
\author{Fátima Brambilla}

% Enable SageTeX to run SageMath code right inside this LaTeX file.
% documentation: http://mirrors.ctan.org/macros/latex/contrib/sagetex/sagetexpackage.pdf
% \usepackage{sagetex}
\begin{document}


\maketitle {}

\section {Introducción}
La atmósfera de la Tierra es una capa de gases, comúnmente conocidas como "aire", que rodea a la Tierra y es retenida en ella debido a la gravedad de la misma. La atmósfera en la Tierra protege la vida en esta, ya que al crear presión sobre el planeta, permite la existencia de agua liquida en la superficie terrestre, así mismo, protege al planeta de la radiación ultravioleta, y mediante un efecto invernadero mantiene la superficie a una temperatura ídonea, reduciendo las temperaturas extremas entre el día y la noche.

\begin{figure}
    \caption{Capa exterior de la Atmósfera}
    \includegraphics[width=0.5\textwidth]{Top_of_Atmosphere.jpg}
    \centering
    \label{Atmósfera}
\end{figure}

\section{Composición}

Los tres mayores componentes del aire, y de la atmósfera terrestre misma, son nitrógeno, oxigeno y argón. El vapor de agua constituye alrededor del 0.25\% de la masa atmósferica. La concentración del vapor de agua varia considerablemente, yendo desde alrededor de 10 ppm del volumen en los puntos más fríos de la atmósfera hasta el 5\% del volumen en puntos calientes, masas de aire de húmedo, y concentraciones de otros gases atmósfericosson llamados con regularidad "aire seco", sin incluir al vapor del agua.
Se suele referir al resto de gases como "gases restantes" o de rastro,entre los cuales se encuentran los gases de efecto invernadero, principalmente el dióxido de carbono,dióxido de nitroso,metano,y ozono.

\section {Estructura de la Atmósfera}
\subsection {Principales capas}
En general, la presión del aire y la densidad decrecen con la incremento en la altura. Sin embargo, es la temperatura quien tiene un comportamiento más complicado con la altura, siendo capaz de mantenerse relativamente constante o de incrementar considerablemente con la altura en algunas regiones.Debido a que en general, el comportamiento altura/temperatura es constante de acuerdo a los sondeos de globo, el comportamiento de la temperatura provee una útil métrica para distinguir las capaz de la atmósfera.
De esta forma, la atmósfera terrestre puede ser dividida en cinco capaz principales, listadas a continuación de la más alta a la más baja

\subsection {Exósfera}
La exósfera es la capa exterior de la atmósfera, la cual se extiende desde la "exóbase", situada al final de la termósfera, a una altura de 700km sobre el nivel del mal, hasta unos 10000km donde se fusiona con los vientos solares
Esta capa esta compuesta mayormente por pequeñas densidades de hidrógeno, helio, y muchas molecúlas pesadas como el nitrógeno, dióxido de carbono y oxigeno cerca de la exóbase. Las molecúlas y átomos están tan separados unos de otros, que pueden viajar miles de kilométros sin siquiera toparse unos con otros, por ello la exósfera no se comporta como un gas, y constantemente sus partículas se escapan al espacio.
La exósfera esta bastante por encima de la tierra como para que cualquier fenómeno meteréologico sea posible, sin embargo, las auroras boreales y austrális de vez en cuando ocurren en la parte más baja de la exósfera, sobreponiéndose a la termósfera

\subsection {Termósfera}
La termósfera es la segunda capa más alta de la atmósfera, se extiende desde la mesopáusa (la cual la separad de la mesósfera) a una altitud aproximadamente de 80 kilométros hasta la termopáusia, a una altitud dentro del rango de 500 a 1000km. La altura de la termopáusa varia considerablemente debido a los cambios en la actividad solar. Ya que la termopáusa se encuentra en la parte más baja de la exósfera, también se le llama exóbase.
La parte más baja de la termósfera, de 80 a 550km sobre la superficie terrestre, contiene la Ionosféra.
La temperatura de la termósfera incrementa gradualmente con el incremento de altura. A diferencia de la estratósfera, debajo de ella, donde la inversión de la temperatura es debida a la radiación por el ozono,la inversión en la termósfera ocurre debido a la extrema baja densidad de las partículas. La temperatura en esta capa puede alcanzar hasta 1500 grados Celsius, aunque las molecúas de gas está tan apartadas de entre ellas, que la temperatura, tomada en el sentido común no es algo muy significativo.

\subsection {Mesósfera}
La mesósfera es la tercera capa más alta, ocupando la región por encima de la estratósfera y por debajo de la termósfera. Se extiende desde la estropáusa a una altura de 50km hasta la mesopáusa a la altura de entre 80 y 85km sobre el nivel del mar. La temperatura decae con la creciente altitud a la mesopáusa que marca el limite de esta capa media de la atmósfera. Esta es la zona más helada de la Tierra, y tiene un promedio de temperatura de menos 85 grados Celsius.
La mesósfera es la capa donde la mayoría de los meteoritos se calcinan al entrar en contacto con la atmósfera. La altura de esta capa es demasiada como para permitir el acceso de aviones de propulsión, y globos. La mesósfera es mayormente accedida por cohetes de sondeo y cohetes de propulsión.

\subsection {Estratósfera}
La estratósfera es la segunda capa más baja de la atmósfera, esta por encima de la trotósfera y es  separada de ella por la trotopáusa. Esta capa se extiende desde el limite superior de la trotósfera a unos 12km sobre la superficie terrestre, hasta la estratopáusa, a una altura entre los 50 y 55km.
La presión atmósferica en la cima de la estratósfera es de 0.001 veces la presión al nivel del mar. Esta capa contiene a la capa de ozono, que es la parte de la atmósfera terrestre que contiene altas concentraciones del gas. La estratófera es una capa donde la temperatura aumenta junto a la altura, esto es causado por la absorción de la radiación de los rayos ultravioletas del sol en la capa de ozono. Aunque la temperatura en la estratopáusa puede ser de unos 60 grados bajo cero, a la altura máxima de a estratósfera es mucho más caliente, estando cerca de los cero grados.

\subsection {Trotósfera}
La trotósfera es la capa más baja, la cual se extiende desde la superficie terrestre hasta alrededor de una altura de 12km, aunque en realidad esta altura puede variar, desde los 9km hasta los 17km en el ecuador, con algunas variaciones debido al clima. Esta capa es limitada por la trotopáusa, un limite que es marcado en algunos lugares por una inversión en la temperatura, y otros por una zona que es isótermica con la altura.
Por supuesto, aun si ocurren variaciones, la temperatura normalmente decrece a medida que aumenta la altura en la trotósfera, ya que esta capa es mayormente calentada por la energía transferida desde la superficie. Por ello, la parte más baja de esta capa es la más caliente.
La trotósfera contiene alrededor del 80\% de la masa de la atmósfera. Esta capa es más densa que el resto juntas, porque el peso atmósferico recae en ella, haciendo que sea la más comprimida.

\subsection {Otras capas}
Entre las principales capas de la atmósfera, que pueden ser distinguidas mediante la temperatura, existen otras que pueden diferenciarse por medio de otras carácteristicas:

\subsubsection {Capa de ozono}
Esta capa esta contenida en la estratósfera, y en ella la concentración de ozono es de alrededor de 2 a 8 partes por millón, lo cual es mucha más concentración que en la parte baja de la atmósfera, aunque sigue siendo una cantidad considerada pequeña si se compará a otros componentes de la misma. Esta capa esta mayormente localizada en la parte baja de la estratósfera, entre los 15 y 35km de altura, aunque su espesor varía conforme a las estaciones y la geografía.

\subsubsection {Ionósfera}
La ionósfera es una región de la atmósfera que esta ionizada por la radiación solar. Esta es responsable por las auroras. Durante las horas del día se estrecha de unos 50 a 1000km,e incluye a la mesósfera, termósfera y parte de la exósfera. No obstante, la ionización en la mesósfera suele cesar en las horas nocturnas, por lo que las auroras solo son visibles en la termósfera y parte de la exósfera.
La ionósfera forma el borde interior de la magnetósfera, la cual tiene una importancia práctica debido a sus influencias, por ejemplo, la propagación de radio en la Tierra.

\subsubsection {Homósfera}
La homósfera y heterósfera son definidas por la combinación de los gases atmósfericos. La base superficial de la homósfera incluye a la trotósfera, estratósfera, mesósfera y la parte baja de la termósfera, donde la composición química de la atmósfera no depende del peso molecular, ya que los gases están mezclados por la turbulencia. Esta capa relativamente homógenea termina en la tubopáusa, la cual esta a 100km aproximadamente, el mismo limite espacial, aceptado por FAI,situado a unos 20km arriba de la mesopáusa.
A esta altura se encuentra la heterósfera, la cual incluye la exósfera y la mayor parte de la termósfera. En ella, la composición química varía con la altura, esto es debido a que la distancia en que las partículas se pueden mover sin colisionar unas con otras es grande comparada con el tamaño de movimientos que causan la mezcla. Esto permite a los gases se estratifiquen por peso molecular, con los más pesados, como el oxígeno y nitrógeno, presentes únicamente en la parte más baja de la heterósfera, siento la parte más alta de esta capa, compuesta casi por completo de hidrógeno, el elemento más ligero.

\subsubsection {Capa limite planetaria}
Esta es la parte de la trotósfera que esta más cercana a la superficie terrestre, y es la más afectada por ella, mayormente por la difusión turbulenta. Durante el día, la capa limite planetaria, por lo usual esta bien mezclada, por otro lado, durante la noche se estratifica de manera estable, o intermitente.
El rango de profundidad de esta capa va desde 100m en claras y tranquilas noches, hasta unos 3000m o más durante la tarde en regiones secas.

\section {Propiedades físicas}
\subsection {Presión y espesor}
La presión atmósferica promedio esta definida por la "International Standar Atmosphere" como 101325 Pascales. Algunas veces se refiere a esto como la unidad estándar de atmósferas (atm). La masa atmósferica total es de $5.1480x10^18$kg, alrededor de 2.5\% menos que sería inferido por el promedio de presión al nivel del mar y el área de la Tierra, de 51007.2 megahectáreas, siendo esta porción desplazada por los terrenos montañosos de la Tierra.
Aun si la presión del aire varía con la altura y el clima, la presión atmósferica es el total del peso del aire sobre unidad de área, hasta el punto en que puede ser mesurado.

\subsection {Temperatura y velocidad del sonido}
La temperatura decrece con el aumento de la altura, siendo así desde el nivel del mar, pero a partir de los 11km empiezan variaciones en esto, donde la temperatura se estabiliza a través de una larga distancia vertical hasta el resto de la trotósfera.
En la estratósfera, a partir de la altura de 20km, la temperatura incrementa con la altura, debido al calentamiento en la capa de ozono, causado por atrapar significativamente la radiación de los rayos ultravioletas del sol, por el dióxigeno y ozono en esta área.
Porque en un gas ideal, de composición constante la velocidad del sonido depende de forma exclusiva de la temperatura, y no de la densidad o la presión del gas, la velocidad del sonido en la atmósfera con la altura, toma un perfil complicado basado en la temperatura, y no refleja cambios altitudinales en presión o densidad.

\subsection {Densidad y masa}
La densidad del aire es de alrededor de $1.2kg/m^3$. La densidad no puede ser medida directamente, sino que es calculada mediante mediciones de temperatura, presión y humedad, usando la ecuación del estado del aire (una forma de la ley de gas ideal). La densidad atmósferica decrese al aumentar la altura. Esta variación puede ser modelada aproximadamente usando la formula barométrica.
En promedio, la masa atmósferica esta cerca a 5 cuatrillones de toneladas, de acuerdo a la "American National Center for Atmospheric Research", la masa total de la atmósfera es $5.1480x10^18kg$ con un rango anual debido al vapor de agua de 1.2 o $1.5x10^15kg$, dependiendo de si se usan datos de presión superficial o vapor de agua.

\section {Propiedades ópticas}
La radiación solar es la energía que recibe la Tierra del Sol. El planeta también emite radiación de regreso al espacio, pero a unas longitudes de onda que nos son imperceptibles. Parte de la radiación entrante y emitida es absorbida o reflectada por la atmósfera.

\subsection {Dispersión}
Cuando la luz pasa por la atmósfera terrestre, los fotones interactúan con ella mediante la dispersión. Si la luz no interactúa con la atmósfera, es llamada radiación directa, y es lo que ves si miras directamente al sol. La radiación indirecta es luz que se ha dispersado en la atmósfera.

\subsection {Absorción}
Diferentes molecúlas absorben diferentes longitudes de onda. Cuando una molecúla absorbe un fotón, aumenta la energía de la molecúla. Esto calienta la atmósfera, pero esta última también se enfría al emitir radiación.
Los espectros de combinación combinados de los gases dejan "ventanas" de baja opacidad, permitiendo la transmición de ciertas bandas de luz.

\subsection {Emisión}
Emisión es lo contrario a absorción, es cuando un objeto emite radiación. Los objetos tienden a emitir cantidad y longitudes de onda de radiación dependiendo de las curvas de emisión de su "cuerpo negro", por lo tanto objetos de calientes tienden a emitir más radiación con longitudes de onda menores.
Debido a su temperatura, la atmósfera terrestre emite radiación infrarroja. Por ejemplo, en noches despejadas, la superficie terrestre se enfría más rápido que en noches nubosas, esto es debido a que las nubes son fuertes absorbentes y emiténtes de radaición infrarroja. Esto es también la razón de porque se vuelve más frió a grandes alturas.
El efecto invernadero esta directamente relacionado a este efecto de absorción y emisión. Algunos gases en la atmósfera absorben y emiten radiación infrarroja, pero no interactúan con la luz solar en el espectro visible.

\subsection {Indicé de refracción}
El indicé de refracción del aire es cercano, pero sigue siendo mayor a 1. Las variaciones sistematicas en el indicé de refracción pueden llevar a la flexión de rayos de luz sobre amplios caminos ópticos.
El indicé de refracción del aire depende de la temperatura, levantando los efectos de refracción cuando el gradiente de la temperatura es grande.

\section {Circulación}
La circulación atmósferica es el movimiento de larga escala del aire que sucede en la trtósfera, por cual el calor es distribuido por la Tierra. La estructura a larga escala de la círculación atmósfera varia de un año a otro, pero la estructura básica permanece constante ya que es determinada por la rotación de la Tierra, y la diferencia de la radiación solar entre los polos y el ecuador.

\begin{figure}
    \caption{Circulación del aire}
    \includegraphics[width=0.5\textwidth]{AtmosphCirc2.png}
    \centering
    \label{Circulacion}
\end{figure}

\section {Bibliografia}
\begin {enumerate}
\item NASA. Gateaway to Astounat Photos of Earth.
\item Zimmer, Carl. Earth Oxygen: A Mistery Easy to Take for Granted. New York Times.
\item Lide, David R. Handbook of Chemistryand Physichs. Boca Raton, FL: CRC, 1996. 14-7

\end{enumerate}


\end{document}
